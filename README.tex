% Options for packages loaded elsewhere
\PassOptionsToPackage{unicode}{hyperref}
\PassOptionsToPackage{hyphens}{url}
%
\documentclass[
]{article}
\usepackage{lmodern}
\usepackage{amssymb,amsmath}
\usepackage{ifxetex,ifluatex}
\ifnum 0\ifxetex 1\fi\ifluatex 1\fi=0 % if pdftex
  \usepackage[T1]{fontenc}
  \usepackage[utf8]{inputenc}
  \usepackage{textcomp} % provide euro and other symbols
\else % if luatex or xetex
  \usepackage{unicode-math}
  \defaultfontfeatures{Scale=MatchLowercase}
  \defaultfontfeatures[\rmfamily]{Ligatures=TeX,Scale=1}
\fi
% Use upquote if available, for straight quotes in verbatim environments
\IfFileExists{upquote.sty}{\usepackage{upquote}}{}
\IfFileExists{microtype.sty}{% use microtype if available
  \usepackage[]{microtype}
  \UseMicrotypeSet[protrusion]{basicmath} % disable protrusion for tt fonts
}{}
\makeatletter
\@ifundefined{KOMAClassName}{% if non-KOMA class
  \IfFileExists{parskip.sty}{%
    \usepackage{parskip}
  }{% else
    \setlength{\parindent}{0pt}
    \setlength{\parskip}{6pt plus 2pt minus 1pt}}
}{% if KOMA class
  \KOMAoptions{parskip=half}}
\makeatother
\usepackage{xcolor}
\IfFileExists{xurl.sty}{\usepackage{xurl}}{} % add URL line breaks if available
\IfFileExists{bookmark.sty}{\usepackage{bookmark}}{\usepackage{hyperref}}
\hypersetup{
  pdftitle={asremlPlus},
  hidelinks,
  pdfcreator={LaTeX via pandoc}}
\urlstyle{same} % disable monospaced font for URLs
\usepackage[margin=1in]{geometry}
\usepackage{graphicx,grffile}
\makeatletter
\def\maxwidth{\ifdim\Gin@nat@width>\linewidth\linewidth\else\Gin@nat@width\fi}
\def\maxheight{\ifdim\Gin@nat@height>\textheight\textheight\else\Gin@nat@height\fi}
\makeatother
% Scale images if necessary, so that they will not overflow the page
% margins by default, and it is still possible to overwrite the defaults
% using explicit options in \includegraphics[width, height, ...]{}
\setkeys{Gin}{width=\maxwidth,height=\maxheight,keepaspectratio}
% Set default figure placement to htbp
\makeatletter
\def\fps@figure{htbp}
\makeatother
\setlength{\emergencystretch}{3em} % prevent overfull lines
\providecommand{\tightlist}{%
  \setlength{\itemsep}{0pt}\setlength{\parskip}{0pt}}
\setcounter{secnumdepth}{-\maxdimen} % remove section numbering

\title{asremlPlus}
\author{}
\date{\vspace{-2.5em}}

\begin{document}
\maketitle

\href{http://www.repostatus.org/\#active}{\includegraphics{http://www.repostatus.org/badges/latest/active.svg}}
\href{https://cran.r-project.org/}{\includegraphics{https://img.shields.io/badge/R\%3E\%3D-2.10.0-6666ff.svg}}
\href{https://cran.r-project.org/package=asremlPlus}{\includegraphics{http://www.r-pkg.org/badges/version/asremlPlus}}
\href{/commits/master}{\includegraphics{https://img.shields.io/badge/Package\%20version-4.2--11-orange.svg?style=flat-square}}
\href{/commits/master}{\includegraphics{https://img.shields.io/badge/last\%20change-2020--02--08-yellowgreen.svg}}
\href{http://choosealicense.com/licenses/mit/}{\includegraphics{https://img.shields.io/github/license/mashape/apistatus.svg}}
\href{commits/master}{\includegraphics{https://cranlogs.r-pkg.org/badges/last-week/asremlPlus}}

\texttt{asremlPlus} is an R package that augments the use of
\texttt{ASReml-R} in fitting mixed models and packages generally in
exploring prediction differences. This version is compatible with both
\texttt{ASReml-R} versions 3 and 4.1, but not 4.0.

Versions 4.x-xx of \texttt{asremlPlus} are a major revamp of the package
and include substantial syntax changes. In particular, most functions
are S3 methods and so the type of the object can be omitted from the
function name when calling the function.

\hypertarget{more-information}{%
\subsection{More information}\label{more-information}}

For more information install the package and run the R command
\texttt{news(package\ =\ “asremlPlus”)} or consult the
\href{./vignettes/asremlPlus-manual.pdf}{manual}.

An overview can be obtained using \texttt{?asremlPlus}. In particular,
an example of its use is given towards the bottom of the help
information and this is avalable as the
\href{./vignettes/Wheat.analysis.pdf}{Wheat.analysis vignette}. It that
shows how to select the terms to be included in a mixed model for an
experiment that involves spatial variation; it also illustrates
diagnostic checking and prediction production and presentation for this
example. A second vignette is the
\href{./vignettes/Wheat.infoCriteria.pdf}{Wheat.infoCriteria vignette}
that illustrates the facilities in \texttt{asremlPlus} for producing and
using information criteria. Two further vignettes show how to use
\texttt{asremlPlus} for exploring and presenting predictions from a
linear mixed model analysis in the context of a three-factor factorial
experiment on ladybirds: one vignette,
\href{./vignettes/Ladybird.asreml.pdf}{Ladybird.asreml vignette}, uses
\texttt{asreml} and \texttt{asremlPlus} to produce and present
predictions; the other vignette,
\href{./vignettes/Ladybird.asreml.pdf}{Ladybird.lm vignette}, uses
\texttt{lm} to produce the predictions and and \texttt{asremlPlus} to
present the predictions. The vignettes can be accessed via
\texttt{vignette(name,\ package\ =\ "asremlPlus")}, where \texttt{name}
is one of \texttt{"Wheat.analysis"}, \texttt{"Wheat.infoCriteria"},
\texttt{"Ladybird.asreml"} or \texttt{"Ladybird.lm"}.

\hypertarget{installing-the-package}{%
\subsection{Installing the package}\label{installing-the-package}}

\hypertarget{from-a-repository-using-drat}{%
\subsubsection{\texorpdfstring{From a repository using
\texttt{drat}}{From a repository using drat}}\label{from-a-repository-using-drat}}

Windows binaries and source tarballs of the latest version of
\texttt{asremlPlus} are available for installation from my
\href{http://chris.brien.name/rpackages}{repository}. Installation
instructions are available there.

\hypertarget{directly-from-github}{%
\subsubsection{Directly from GitHub}\label{directly-from-github}}

\texttt{asremlPlus} is an R package available on GitHub, so it can be
installed from the RStudio console or an R command line session using
the \texttt{devtools} command \texttt{install\_github}. First, make sure
\texttt{devtools} is installed, which, if you do not have it, can be
done as follows:

\texttt{install.packages("devtools")}

Next, install \texttt{asremlPlus} from GitHub by entering:

\texttt{devtools::install\_github("briencj/asremlPlus")}.

Version 2.0-12 of the package is available from CRAN so that you could
first install it and its dependencies using:

\texttt{install.packages("asremlPlus")}

If you have not previously installed \texttt{asremlPlus} then you could
first install it and its dependencies from CRAN using:

\texttt{install.packages("asremlPlus")}

Otherwise, you will need to install its dependencies manually:

\texttt{install.packages(c("dae",\ "ggplot2",\ "reshape",\ "plyr",\ "dplyr",\ "stringr",\ "RColorBrewer",}
\texttt{"foreach",\ "parallel",\ "doParallel"))}

\hypertarget{what-is-does}{%
\subsection{What is does}\label{what-is-does}}

It assists in automating the testing of terms in mixed models when
\texttt{asreml-R} is used to fit the models. A history of the fitting of
a sequence of models is kept in a data frame. Procedures are available
for choosing models that conform to the hierarchy or marginality
principle. It can also be used to display, in tables and graphs,
predictions obtained from a mixed model using your favourite model
fitting functions and to explore differences between predictions. As a
general rule functions that are methods for \texttt{asreml} and
\texttt{asrtests} objects require \texttt{asreml-R}; on the other hand,
functions that are methods for \texttt{alldiffs} and \texttt{data.frame}
objects do not require \texttt{asreml-R}.

The use of the package is exemplified in three vignettes: the
\href{./vignettes/Wheat.pdf}{Wheat vignette} that is accesssed using
\texttt{vignette("Wheat",\ package\ =\ "asremlPlus")}, shows how to
select the terms to be included in a mixed model; two further vignettes
show how to use \texttt{asremlPlus} for exploring and presenting
predictions: one vignette
\href{./vignettes/Ladybird.asreml.pdf}{Ladybird asreml vignette} uses
\texttt{asreml} to produce the predictions and is available via
\texttt{vignette("Ladybird.asreml",\ package\ =\ "asremlPlus")}; the
other vignette \href{./vignettes/Ladybird.asreml.pdf}{Ladybird lm
vignette} uses \texttt{lm} to produce the predictions and is available
via \texttt{vignette("Ladybird.lm",\ package\ =\ "asremlPlus")}.

The content falls into the following natural groupings:

\begin{enumerate}
\def\labelenumi{(\roman{enumi})}
\item
  Data,
\item
  Object manipulation functions,
\item
  Model modification functions,
\item
  Model testing functions,
\item
  Model diagnostics functions,
\item
  Prediction production and presentation functions,
\item
  Response transformation functions, and
\item
  Miscellaneous functions.
\end{enumerate}

For a list of the functions for each group, see the help for
\texttt{asremlPlus-package} or the entry in the manual for
\texttt{asremlPlus-package}.

\hypertarget{what-it-needs}{%
\subsection{What it needs}\label{what-it-needs}}

To use those functon in \texttt{asremlPlus} that are methods for
\texttt{asreml} or \texttt{asrtests} objects, you must have a licensed
version of the package \texttt{asreml}. It provides a computationally
efficient algorithm for fitting mixed models using Residual Maximum
Likelihood. It can be purchased from `VSNi' \url{http://www.vsni.co.uk/}
as \texttt{asreml-R}, who will supply a zip file for local
installation/updating.

It also imports \href{https://CRAN.R-project.org/package=dae}{dae},
\href{https://CRAN.R-project.org/package=ggplot2}{ggplot2},
\texttt{stats}, \texttt{methods}, \texttt{utils},
\href{https://CRAN.R-project.org/package=reshape}{reshape},
\href{https://CRAN.R-project.org/package=plyr}{plyr},
\href{https://CRAN.R-project.org/package=dplyr}{dplyr},
\href{https://CRAN.R-project.org/package=stringr}{stringr},
\href{https://CRAN.R-project.org/package=RColorBrewer}{RColorBrewer},
\texttt{grDevices},
\href{https://CRAN.R-project.org/package=foreach}{foreach},
\href{https://CRAN.R-project.org/package=parallel}{parallel},
\href{https://CRAN.R-project.org/package=doParallel}{doParallel}.

\hypertarget{license}{%
\subsection{License}\label{license}}

The \texttt{asremlPlus} package is distributed under the
\href{https://opensource.org/licenses/MIT}{MIT licence} -- for details
see
\href{https://github.com/briencj/asremlPlus/blob/master/LICENSE.md}{LICENSE.md}.

\end{document}
